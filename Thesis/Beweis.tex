\section{proof of the main theorem}
\subsection{(7.1) bis Step 2}
$S = \Vp(f) \subset \proj^3$ sei eine nicht-singuläre Kubik mit $f \in k[x,y,z,t]$.

\subsubsection{ %%%%%%%%%%%%%%%%%%%%%%%%%%%%%%%%%%%%%%%
Behauptung: $l =  T_P(l) \subset T_P(S)$
} %%%%%%%%%%%%%%%%%%%%%%%%%%%%%%%%%%%%%%%%%%%%%

Wir erinnern uns an die Definition des Tangentialraumes $T_P(\Vp(I)) = \cap_{f\in I} \Vp(f_P)$.
Weiters sei $l = \Vp(H_1, H_2) \subset S := \Vp(f)$ die Gerade auf der Kubik.
Dann erhalten wir $l = \Vp(H_1) \cap \Vp(H_2) = \cap_{\alpha, \beta} \Vp(\alpha H_1 + \beta H_2) \defeq  T_P(l)$.
Die zweite Gleichung folgt aus der Tatsache, dass $\Ip(l) \supset \Ip(S)$, der Schnitt also über eine größere Menge stattfindet.


\subsubsection{%%%%%%%%%%%%%%%%%%%%%%%%%%%%%%%%%%%%%%%
Behauptung: Sei $\Vp(H) \subset \proj^3$ eine Ebene.
% TODO:
(Diese Ebene soll nicht in der Kubik enthalten sein).
Es gibt eine 3-Form $h \in k[x,y,z]$ und eine Inklusion $\iota : \proj^2 \hookr \proj^3$ sodass $\Vp(H) = \proj^2$ und $S\cap \proj^2 = \Vp(h) \subset \proj^2$.
} %%%%%%%%%%%%%%%%%%%%%%%%%%%%%%%%%%%%%%%%%%%%%

oBdA sei $H = t - \alpha x -  \beta y - \gamma z$ mit $\beta, \gamma \in k$.

Mit homogenen Koordinaten:
\paragraph{(1) $\Vp(H)$ ist isomorph zu $\proj^2$.}
Ein Punkt $[x_0:x_1:x_2:x_3]$ liegt auf $\Vp(H)$ genau dann, wenn $x_3 = \alpha x_0 + \beta x_1 + \gamma x_2$.
Hier kann man auch anmerken, dass $x_0,x_1,x_2$ nie gleichzeitig verschwinden können, da sonst auch $x_3 = 0$ folgen würde.
Damit ist die Projektion $\pi : [x_0:x_1:x_2:x_3] \mapsto [x_0:x_1:x_2]$ wohldefiniert.
Es liegt nahe, eine Inverse $\proj^2 \to \Vp(H)$ wie folgt zu definieren:
$\pi^{-1} : [x_0:x_1:x_2] \mapsto [x_0 : x_1 : x_2 : \alpha x_0 + \beta x_1 + \gamma x_2]$.
Die Abbildung ist wohldefiniert, denn $x_1,x_2$ verschwinden nicht simultan und $H(\pi^{-1}([x_0:x_1:x_2])) = 0$.

\paragraph{(2) var{Der} Schnitt $\Vp(H) \cap S$ wird via $\pi$ auf eine Varietät $\Vp(h) \subset \proj^2$ transportiert, wobei $h$ eine 3-Form ist.}

Sei $[x_0:x_1:x_2] \in \proj^2$ ein Punkt.
$[x_0:x_1:x_2] \in \pi(\Vp(H)\cap S)
\Leftrightarrow f(\pi^{-1}([x_0:x_1:x_2])) = 0
\Leftrightarrow f(x_0,x_1,x_2,\alpha x_0 + \beta x_1 + \gamma x_3) = 0$, d.h. $\pi(\Vp(H)\cap S) = \Vp(g)$ mit $g=f(x, y, z, \alpha x + \beta y + \gamma z) = \var{eval}(\var{---},(x, y, z, \alpha x + \beta y + \gamma z))(f)$ eine 3-Form.

\subsubsection{%%%%%%%%%%%%%%%%%%%%%%%%%%%%%%%%%%%%%%%
Behauptung: Es gibt eine Darstellung von $f = h + HB$ mit $h$ die obige "Restriktion" von $f$ auf $\Vp(H)$ sowie $B$ eine 2-Form.
} %%%%%%%%%%%%%%%%%%%%%%%%%%%%%%%%%%%%%%%%%%%%%


Wir können die Auswertung an $\theta := (x,y,z,\alpha x + \beta y + \gamma z)$ ergänzen zu einem var{Hom}omorphismus $k[x,y,z,t] \to k[x,y,z] \hookr k[x,y,z,t]$ mit Kern $H$. Weiters gilt $\var{eval}(\var{---},\theta)(f) = \var{eval}(\var{---},\theta).(\var{eval}(\var{---},\theta))(f)$, also $f = \var{eval}(f,\theta) + p = g+p$ wobei $p \in \ker(\var{eval}(\var{---},\theta)) = (H)$.
Es gibt also eine Darstellung $p = HB$ für ein Polynom $B$.
Nun nutzen wir aus, dass $f$ eine $3-Form$ ist und erhalten $f = g+HB$ wobei $B$ eine quadratische Form ist.

\subsubsection{%%%%%%%%%%%%%%%%%%%%%%%%%%%%%%%%%%%%%%%
Behauptung: Für $f = h + HB$ wie oben hat $h$ nicht die Form $g^2A$ mit $g,A$ 1-Formen.
} %%%%%%%%%%%%%%%%%%%%%%%%%%%%%%%%%%%%%%%%%%%%%

Angenommen dies wäre der Fall, also $f = g^2A+HB$.
Wir wollen zeigen, dass dann ein singulärer Punkt auf $S$ existiert.
Dieser erfüllt genau $f_x = f_y = f_z = f_t = 0$ (vgl. Shafarevich, BAG1).
Wir brauchen eine verallgemeinerte Aussage.



\subsubsection{LEMMA 1, LEMMA 1b, LEMMA 2}%%%%%%%%%%%%%%%%%%%%%%%%%%%%%%%%%%%%%%%
LEMMA 1: Sei $R$ eine $k$-Algebra erzeugt durch $y_0,..y_n$ algebraisch unabhängig.
Dann gibt es var{Der}ivationen $D_i : R \to k$ sodass $D_i(y_i) = 1$ und $D_i(y_j) = 0$ für $j\neq i$. Bezeichne diese var{Der}ivationen als "partielle Ableitungen".

LEMMA 1b: Angenommen $V = \var{span}(y_i) = \var{span}(y'_i)$ als $k$-Vektorräume, dann gibt es eine Basiswechselmatrix $M$ mit $(\vec y') = M(\vec y)$ und $\vec Df = M \vec Df'$.

LEMMA 2:
Sei $f$ eine $d$-Form und $S = \Vp(f)$ eine Hyperfläche. Die singulären Punkte von $S$ sind gegeben durch $S^{sing} = \Vp(f_{y_0},..f_{y_n})$. Hier bezeichnet $f_{y_i}$ die Ableitung von $f$ nach $y_i$, also $f_{y_i} := D_i(f)$.
%%%%%%%%%%%%%%%%%%%%%%%%%%%%%%%%%%%%%%%%%%%%%

\paragraph{(1)}
Siehe (Matsumura, 26.F).

\paragraph{(1b)}
Ich behaupte es gibt einen Iso von $k$-Vektorräumen $\var{Hom}_k(V,k) \overset\sim\to \var{Der}_k(R,k)$.
Weiters induziert $M$ als lineare Abbildung $V \to V$ einen Basiswechsel der Dualbasen: $\vec y'^* = M(\vec y^*)$.

\paragraph{(2)} 
Sei $R = k[x_0,..x_n]$ ein Polynomring über einem Körper $k$ und $d$ eine natürliche Zahl mit $char(k)$ teilt nicht $d$.
Seien weiters $y_0,..y_n \in R$ eine Basis der 1-Formen (notwendigerweise sind die $y_i$ linear).
Seien weiters $y'_0,..y'_n \in R$ eine Basis der 1-Formen (notwendigerweise sind die $y'_i$ linear).
Anwendung von LEMMA 1 gibt und partielle Ableitungen $D_i, D'_i$.
Ich behaupte, dass ein Punkt ist singulär bzgl der $D_i$, genau dann, wenn er singulär bzgl der $D'_i$ ist.
Dies folgt direkt aus LEMMA 1b.
LEMMA 2 folgt dann aus (Shafarevich, BAG1) mit $y'_i = x_i$.


\paragraph{Weiter im Beweis}
Da $f$ nicht komplett auf der Ebene $\Vp(H)$ verschwindet, sind $g$ und $H$ teilerfremd, bzw. $k$-linear unabhängig. Wir ergänzen zu einer Basis von $\var{span}_k(x,y,z,t) = \var{span}(H,g,r_1,r_2)$.
Dann gelten
\begin{align*}
D_{r_i}(f) =& g^2 D_{r_i}(A)  + H D_{r_i}(B) \\
D_H(f)     =& g^2 D_H(A)      + H D_H(B) + B \\
D_g(f)     =& 2gA + g^2D_g(A) + H D_g(B)
\end{align*}

Wir schränken die Ebene $\Vp(H)$ weiter ein auf die Gerade $\Vp(H,g)$, sodass die singulären Punkte von $S$ auf der Geraden definiert sind durch $B = 0$.
Da $B$ aber entweder auf der geraden verschwindet oder eine 2-Form auf ihr ist, hat sie für $k$ algebraisch abgeschlossen Nullstellen und es existieren somit singuläre Punkte, konträr zur Behauptung.


\subsubsection{ %%%%%%%%%%%%%%%%%%%%%%%%%%%%%%%%%%%%%%%
Behauptung: Sei $C$ der Schnitt von $P \in S$ nicht-singulär mit der Tangentialebene $T_P(S)$, $S = \Vp(f)$ irreduzibel und nicht die Ebene $C$ selbst, dann ist $C$ singulär bei $P$.
} %%%%%%%%%%%%%%%%%%%%%%%%%%%%%%%%%%%%%%%%%%%%%

Sei $H = f_x(P)x + f_y(P)y + f_z(P)z + f_t(P)t$ die Gleichung der Tangentialebene und oBdA $f_t(P) \neq 0$ ($P$ war als nicht-singulär angenommen).
Wir restringieren $H$ via $\var{eval}(\var{---}) := \var{eval}(\var{---},(x,y,z,-\frac{f_x(P)}{f_t(P)}x-\frac{f_y(P)}{f_t(P)}y-\frac{f_z(P)}{f_t(P)}z))$ und erhalten
$f' = \var{eval}(f)$. Demonstrativ leite ich nach $x$ ab, aber dieselbe Rechnung funktioniert aus Symmetriegründen natürlich auch mit $y,z$:
\begin{equation}
f'_x = f_x D_xx + f_y D_x y + f_z D_x z + f_t D_x (-\frac{f_x(P)}{f_t(P)}x-\frac{f_y(P)}{f_t(P)}y-\frac{f_z(P)}{f_t(P)}z) = f_x - \frac{f_x(P)}{f_t(P)} f_t
\end{equation}
Offensichtlich verschwindet $f'_x$ in $P$: $f'_x(P) = f_x(P) - \frac{f_x(P)}{f_t(P)} f_t(P) = 0$.
\subsection{(7.1) bis (7.2) Step 2} .
\subsection{(7.2) Step 3} .
\subsection{(7.2) Step 4}

\subsubsection{ %%%%%%%%%%%%%%%%%%%%%%%%%%%%%%%%%%%%%%%
Behauptung: Sei $M \in R^{n\times n}$ eine Matrix über dem Polynomring $R$.
Es gilt $\det(\var{lt}(M)) = \var{lt}(\det(M))$ g.d.w. $\det(\var{lt}(M)) = \var{lt}(\det(\var{lt}(M)))$.
} %%%%%%%%%%%%%%%%%%%%%%%%%%%%%%%%%%%%%%%%%%%%%

Das bedeutet natürlich, dass wir nur $\var{lt}(\det(\var{lt}(M))) = \var{lt}(\det(M))$ zeigen brauchen.
Lüge: Summen $x+y$ und Produkte $xy$ in $R$ vertauschen im Folgenden Sinne mit der $\var{lt}$-Operation: $\var{lt}(x+y) = \var{lt}(\var{lt}(x) + \var{lt}(y))$ und $\var{lt}(xy) = \var{lt}(\var{lt}(x)\var{lt}(y))$.
Die Leibnitzformel liefert nun:
\begin{align*}
  \var{lt}(\det(M))
&= &\var{lt}\left(\sum_\sigma (-1)^\sigma \prod_i M_{i,\sigma(i)}\right)
&= &\var{lt}\left(\sum_\sigma \var{lt}\left((-1)^\sigma \prod_i M_{i,\sigma(i)}\right)\right)
& &
\\
&= &\var{lt}\left(\sum_\sigma \var{lt}\left((-1)^\sigma \prod_i \var{lt}(M_{i,\sigma(i)})\right)\right)
&= &\var{lt}\left(\sum_\sigma (-1)^\sigma \prod_i \var{lt}(M_{i,\sigma(i)})\right)
&= &\var{lt}\left(\det(\var{lt}(M))\right)
\end{align*}
