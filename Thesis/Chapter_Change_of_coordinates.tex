\section{Change of coordinates}

\subsection{Notes to myself}

In this section I introduce definitions and hands-on examples to work with geometric objects in the language of scheme theory, or maybe I'll just drop the generalities and work with elementary algebra.
Particular goals are
\begin{itemize}
\item the definition of what a change of coordinates means precisely and work out the details to use it without regret in the proof of the main result.
\item the definition of what it means for a variety to contain a doubled line. In scheme theoretic language it is as simple as saying, that this variety has a subscheme isomorphic to a doubled line! Again, I might drop the notion of schemes, or work with it.
\end{itemize}
So this text also aims to be educational and close the gap between the intuitive and the technical.

It might be interesting to
\begin{itemize}
\item introduce the category of varieties and rational functions. \textbf{If needed}.
\end{itemize}

\subsection{Affine linear transformations of affine space}

An important notion which will appear again and again is the change of coordinates.
In affine space one thinks of a linear transformation $T:\affine^n_k \to \affine^n_k$.
Now observe the effect of such a transformation on a hypersurface $V(I)$, $I = (f)$:
$ T(V(I)) = \{ Tx \in k^n: f(x) = 0 \} = \{y \in k^n : f(T^{-1}y) = 0\}$
So one may define a linear transformation on a variety by precomposing the inverse to the polynomial functions.
But let me suggest a different definition of linear transformation which does not necessarily require $T$ to be invertible and which is compatible with the scheme theoretic notion of a morphism.

\begin{definition} Let $A = (a_{ij}) \in M(n,k)$ be a $n\times n$-matrix. Then we can define a homomorphism of polynomial rings
\begin{align}
T^\#_A : k[X_1,.. X_n] &\to k[X_1,.. X_n], \\
X_i &\to \sum_{j=1}^n a_{ij}X_j.
\end{align}
\end{definition}

This induces a morphism $T_A : \affine^n_k \to \affine^n_k$ of affine schemes. Indeed, this definition does what we expect.

\begin{proposition} By Hilbert's Nullstellensatz a point $(x_1,..x_n) \in k^n$ corresponds to a maximal ideal $(X_1-x_1,.. X_n-x_n) \subset k[X_1,..X_n]$.
$T_A$ maps the subscheme $V(X_1-x_1,..X_n-x_n)$ to $V(X_1-y_1,..X_n-y_n)$ where $(y_1,..y_n) = A(x_1,..x_n)$.
\end{proposition}

\begin{proof} If $A$ is invertible, we basically apply the Gauß-Jordan algorithm....
\end{proof}

Don't forget the translations: $X_i \mapsto X_i + t_i$.
Together we get affine linear transformations.


\subsection{Rational maps}

Fancy category language here...
