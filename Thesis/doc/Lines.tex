\subsection{lines on surfaces}

% TODO: through two points goes a line
% TODO: How to see if a line through two points lies on a surface

So far we have defined a line to be an intersection of hyperplanes, but of course we also want to understand a line as the unique intersection of such hyperplanes containing two distinct points.

Let's fix a projective space $\proj^n_k$ and let $P=[p_0:..p_n]$ and $Q=[q_0:..q_n]$ be distinct points in $\proj^n_k$. Now consider the homomorphism
\begin{equation}
L : \begin{cases}
k[x_0,..x_n] &\to k[\lambda,\mu] \\
f &\mapsto f(\lambda P + \mu Q) := f(\lambda p_0 + \mu q_0, .. \lambda p_n + \mu q_n)
\end{cases}
\end{equation}

Its kernel contains those polynomials, which vanish on $\lambda P + \mu Q$ and in particular on any point $\lambda_0 P + \mu_0 Q$ for $[\lambda_0:\mu_0] \in \proj^1_k$, such as $P$ and $Q$.
These were the points we expected to be contained on the line anyhow.
Consider the linear map $\bigoplus_{i=0}^n kx_i \to k\lambda \oplus k\mu$ defined by the $2\times (n+1)$-matrix
\begin{equation}
M=
\begin{pmatrix}
p_0 & \ldots & p_n \\
q_0 & \ldots & q_n
\end{pmatrix}
\end{equation}

Because $P$ and $Q$ are distinct points in projective space, the matrix has full rank 2, and hence the kernel is spanned by $(n-1)$ linear forms $h_0,..h_{n-2}$.
By our choice of the matrix $M$, these linear forms lie in the kernel of $L$: $h_i(\lambda P + \mu Q) = h_i(P) \lambda + h_i(Q)\mu = 0 \lambda + 0 \mu$.
In fact, the kernel of $L$ is spanned by these linear forms already, which follows from the \emph{projective Nullstellensatz} which I will just mention here without proof:

\begin{todo}
\item put the Nullstellensatz into the introductory section for projective space, because I reference it before this
\end{todo}

\begin{theorem}[Projective Nullstellensatz]
For a set $X$ of points in projective space $\proj^n_k$ define its ideal $\I(X) := \mkset{f \in k[x_0,..x_n]}{f(x_0,..x_n) = 0\,\forall [x_0:..x_n] \in X}$.
Then $\I$ and $\V$ are inclusion reversing and for ideals $J$ with $\V(J) \neq \emptyset$ we have $\I(\V(J)) = \sqrt{J}$.
\end{theorem}

Assume now that $f \in \ker(L)$, then $\V(f) \supset \mkset{\lambda P + \mu Q}{\lambda, \mu \in k} = \V(h_0,..h_{n-2})$ and hence $(f) \subset \sqrt{(f)} \subset \sqrt{(h_0,..h_n)} = (h_0,..h_n)$.

% TODO
\begin{todo}
\item Show that $\V(h_0,..h_{n-2}) = \{ \lambda P + \mu Q \}$
\end{todo}


