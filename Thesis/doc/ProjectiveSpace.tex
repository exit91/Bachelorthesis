\subsection{projective space}

In order to do geometry we need to explain what points, lines, surfaces are, and this can be done in several ways, synthetically or analytically.

There are several formulations of projective space, with varying degree of generality, and for our purposes I have chosen the one in terms of homogeneous coordinates.

Let $k$ be an algebraically closed field and $n$ a natural number.
The projective space $\proj^n_k$ is a topological space given as set by  $k^{n+1} - \{ 0 \}$ modulo the relation $X=(x_0,..x_n) \sim Y=(y_0,..y_n)$ if and only if $X$ and $Y$ are linearly dependent (as vectors in $k^{n+1}$). The equivalence class of $X$ will be denoted $[x_0:x_1:..:x_n]$.

The topology is given by the closed subsets
\begin{equation}
\V(I) =
\mkset{ [x_0:..x_n] \in \proj^n_k}
      { \forall f\in I \text{ homogeneous}, f(x_0,..x_n) = 0}
\end{equation}
for homogeneous ideals $I$ of the polynomial ring $k[x_0,..x_n]$.
The topology is known as Zariski's topology.
Because $k[x_0,..x_n]$ is a Noetherian ring, every ideal is finitely generated and in particular every homogeneous ideal is finitely generated by homogeneous elements.
So we may say that a closed set consists precisely of the points in projective space where a finite set of homogeneous ideals vanish.

In this framework I want to identify geometric objects with homogeneous ideals of $k[x_0,..x_n]$.
However, most of the time it is easier to speak of the closed subsets of projective space, instead of the ideal $I$, and $I$ itself can be recovered by means of Hilberts's Nullstellensatz in the form of its radical.

A \emph{hypersurface} is given by one equation $f=0$ for $f\in k[x_0,..x_n]$, and its set of points is $\V(f)$.
In case of $f$ being a linear form we call $\V(f)$ a \emph{hyperplane}.

A line is determined by $n-1$ $k$-linearly independent linear forms and a point by $n$ $k$-linearly independent linear forms.



\subsubsection{linear sets}
\begin{todo}
\item define hyperplane, plane, line, point
\item closed subsets yield reduced ideals, but we may attach non-reduced ideals. brought to its full conclusion, we need schemes, but we can do without so far
\item MOVED FROM CONIC SECTION: Of course, in the projective plane there always exists an intersection for any two lines.
% TODO: move this to the front
(This is just a consequence of linear algebra: Say $h_1 = a_0x_0 + a_1x_1 + a_2x_2, h_2 = b_0x_0 + b_1x_1 + b_2x_2$, then the kernel of the matrix $\begin{pmatrix} a_0 & a_1 & a_2 \\ b_0 & b_1 & b_2 \end{pmatrix}$ is nontrivial and if $(s_0,s_1,s_2) \neq 0 \in k^3$ lies in the kernel, then $P := [s_0:s_1:s_2]$ lies in the intersection.)


\end{todo}

\subsubsection{Hilbert's Nullstellensatz}

\begin{todo}
\item put the projective nullstellensatz here
\item corollary for radical $f$: $\V(f) \subset \V(f_0,..f_r) \implies \sqrt{(f_0,..fr)} \subset \sqrt{(f)} \implies f \text{ divides all } f_i$
\item in ptic. if $f$ is a 1-form or a non-singular cubic (as to be seen later)
\end{todo}
