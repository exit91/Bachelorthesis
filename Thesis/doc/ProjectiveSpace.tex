\subsection{Projective space}

In order to do geometry we need to explain what points, lines, surfaces are.
There are several formulations of projective space, with varying degree of generality, and for our purposes I have chosen the one in terms of homogeneous coordinates.

Let $k$ be an algebraically closed field and $n$ a natural number.
Throughout this thesis, $k$ will always be an algebraically closed field.
The projective space $\proj^n_k$ is a topological space given as set by  $k^{n+1} - \{ 0 \}$ modulo the relation $X=(x_0,..x_n) \sim Y=(y_0,..y_n)$ if and only if $X$ and $Y$ are linearly dependent (as vectors in $k^{n+1}$). The equivalence class of $X$ will be denoted $[x_0:x_1:..:x_n]$.

The topology is given by the closed subsets
\begin{equation}
\V(I) =
\mkset{ [x_0:..x_n] \in \proj^n_k}
      { \forall f\in I \text{ homogeneous}, f(x_0,..x_n) = 0}
\end{equation}
for homogeneous ideals $I$ of the polynomial ring $k[x_0,..x_n]$.
The topology is known as Zariski's topology.
Because $k[x_0,..x_n]$ is a Noetherian ring, every ideal is finitely generated and in particular every homogeneous ideal is finitely generated by homogeneous elements.
So we may say that a closed set consists precisely of the points in projective space where a finite set of homogeneous ideals vanish.

In this framework we want to identify geometric objects with homogeneous ideals of $k[x_0,..x_n]$.
However, most of the time it is more natural to speak of the closed subsets of projective space, instead of the ideal $I$, and $I$ itself can be recovered as radical $\sqrt{I}$ by means of Hilbert's Nullstellensatz.
We give the projective version of the Nullstellensatz, as stated in \cite[section 5.3]{reid1988undergraduate}:

\begin{theorem}[Projective Nullstellensatz]
For a set $X$ of points in projective space $\proj^n_k$ define its ideal $\I(X) := \mkset{f \in k[x_0,..x_n]}{f(x_0,..x_n) = 0\,\forall [x_0:..x_n] \in X}$.
Then $\I$ and $\V$ are inclusion reversing and for ideals $J$ with $\V(J) \neq \emptyset$ we have $\I(\V(J)) = \sqrt{J}$.
$\V(J) = \emptyset$ holds iff $J \supset (x_0,..x_n)$.
\end{theorem}

We will now clarify what we mean by a geometric object.
\begin{definition}
A \emph{projective variety} $X$ is a closed subset of some projective space $\proj^n_k$ coming with an ideal $I$, such that $\V(I) = X$.
If the ideal $I$ is not specified, then it will be assumed to be $I=\I(X)$.
\end{definition}
Having defined what objects we consider, we need to say a word about what morphisms are.
\begin{definition}
Let $d \in \naturals$.
A morphism $\phi : X \to Y$ of projective varieties $X \subset \proj^n_k$, $Y \subset \proj^m_k$ is a function between sets $X$ and $Y$,
given by homogeneous polynomials $f_0,..f_m \in k[x_0,..x_n]$ of degree $d$ or 0, that is, $\phi([a_0:..a_n]) = [f_0(a_0,..a_n):..f_m(a_0,..a_n)]$.
We also demand the $f_i$ to have no common zero. \emph{Notation:} We can also write $\phi = (f_0,..f_m)$ as shorthand.
\end{definition}
Notice that such morphisms are continuous.
For more details refer to \cite[part I, lecture 1]{harris1992algebraic}.

\begin{example}
A \emph{hypersurface} is given by one equation $f=0$ for $f\in k[x_0,..x_n]$, and its set of points is $\V(f)$.
In case of $f$ being a linear form we call $\V(f)$ a \emph{hyperplane}.
A line is determined by $n-1$ $k$-linearly independent linear forms and a point by $n$ $k$-linearly independent linear forms.
As a set, a point indeed consists of precisely one point.
Intersections of hyperplanes are called \emph{linear spaces} \cite[example 1.1]{harris1992algebraic}.
\end{example}

\begin{proposition} $n$ hyperplanes $\V(h_1),..\V(h_n)$ in $\proj^n_k$ have precisely at least one common point of intersection.
If the $h_i$ are $k$-linearly independent, then the intersection of all the planes consists of precisely one point.
\end{proposition}
\begin{proof}
To find a point lying on all $n$ hyperplanes amounts to solving a homogeneous system of $n$ linear equations in $n+1$ variables.
Such a system always has at least one non-zero solution.
The $h_i$ being linearly independent implies that the system of equations has full rank, so the solutions have dimension $(n+1) - n = 1$.
\end{proof}

\begin{corollary} \label{corollarySimpleIntersect}
Setting $n=2$ an immediate consequence is that two lines in $\proj^2_k$ intersect.
Because lines are intersections of $n-1$ hyperplanes in $\proj^n_k$, we also deduce that lines always intersect with hyperplanes.
\end{corollary}

So far we have defined a line to be an intersection of hyperplanes, but of course we also want to understand a line as the unique intersection of such hyperplanes containing two distinct points.
Let $P = [p_0:..p_n],Q=[q_0:..q_n] \in \proj^n_k$ be points and consider the vector space $S_1(P,Q)$ of all linear forms vanishing on $P,Q$.
The vector space of all linear forms has dimension $n+1$ and the condition to vanish on $P$ and $Q$ imposes two linear constraints.
The fact that $P\neq Q$ implies that $\dim(S_1(P,Q)) = (n+1) - 2 = n-1$.
Hence $\V(S_1(P,Q)$ gives us the unique line vanishing on $P$ and $Q$.
Any linear form $h$ vanishing on $P$ and $Q$ must vanish on $\lambda P + \mu Q$ by linearity, i.e. $h(\lambda P + \mu Q) = \lambda h(P) + \mu h(Q) = 0$.
As sets, $\V(S_1(P,Q)) = \mkset{\lambda P + \mu Q}{[\lambda:\mu] \in \proj^1_k} = \var{im}(\phi)$, the image of $\phi : \proj^1_k \to \proj^n_k, [\lambda:\mu] \mapsto [\lambda p_0+ \mu q_0:..\lambda p_n + \mu q_n]$.
This can be seen either by linear algebra, or from the fact that the image of $\phi$ is a closed set \cite[theorem 1.10]{shafarevich1994basic} of dimension $> 0$ contained in $\V(S_1(P,Q)$ an irreducible projective variety of dimension $1$.
By \cite[theorem 1.19]{shafarevich1994basic}, $\phi(X) = \V(S_1(P,Q))$.

%%We can obtain these results resorting to pure linear algebra, but to cut this short, we will use some more advanced concepts in the following discussion.

%%Let's fix a projective space $\proj^n_k$ ($n\geq 1$) and let $P=[p_0:..p_n]$ and $Q=[q_0:..q_n]$ be distinct points in $\proj^n_k$.
%%For two projective varieties $X$,$Y$ any morphism $X \to Y$ has a closed image (\cite[theorem 1.10]{shafarevich1994basic}).
%%Clearly, the set $\{ \lambda P + \mu Q \}$ is the image of $\phi : X=\proj^1_k \to Y=\proj^n_k, [\lambda:\mu] \mapsto [\lambda p_0 + \mu q_0,..]$ hence it is closed.
%%Furthermore $\phi(X)$ cannot have dimension\footnote{for a definition of dimension refer to \cite[chapter 6]{shafarevich1994basic}} more than $\dim(\proj^1_k) = 1$.
%%One can see that $\phi(X)$ does not have dimension 0, having infinitely many points, do the dimension indeed is one.
%%To show that it is indeed a linear subspace, consider the vector space $S_1(P,Q)$ of all linear forms vanishing on $P$ and $Q$.
%%It has dimension $(n+1) - 2 = n-1$ and every $h \in S_1(P,Q)$ vanishes on all of $\phi(X)$ as $h(\lambda P + \mu Q) = \lambda h(P) + \mu h(Q) = 0$.
%%So $\phi(X)$ is contained in the line $\V(S_1(P,Q))$ of dimension 1, so by \cite[theorem 1.19]{shafarevich1994basic} $\phi(X) = \V(S_1(P,Q))$.

An easy corollary is
\begin{lemma} \label{lemmaLineOnSurface}
A hypersurface $\V(f)$ containing two points $P$ and $Q$ also contains the line $\overline{P,Q}$ if $f(\lambda P +\mu Q) = 0 \in k[\lambda,\mu]$.
\end{lemma}

If one wishes to, one can show by a completely analogous argument, that a plane is determined by three points $P,Q,R$ on it, such that they don't all lie on a line.

\begin{remark}
Why are we working over an algebraically closed field?
It is known that the projective variety $\V(x^n + y^n - z^n), \,(n>0)$ has no rational points (I won't give a proof).
An even better known fact: $x^2 + z^2 = 0$ has no non-zero solutions over the reals.
To work comfortably we always want to have sufficiently many points on our varieties to work with and Hilbert's Nullstellensatz gives us this guarantee.
\end{remark}
