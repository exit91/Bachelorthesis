\subsection{projective space}

In order to do geometry we need to explain what points, lines, surfaces are, and this can be done in several ways, synthetically or analytically.

There are several formulations of projective space, with varying degree of generality, and for our purposes I have chosen the one in terms of homogeneous coordinates.

Let $k$ be an algebraically closed field and $n$ a natural number.
The projective space $\proj^n_k$ is a topological space given as set by  $k^{n+1} - \{ 0 \}$ modulo the relation $X=(x_0,..x_n) \sim Y=(y_0,..y_n)$ if and only if $X$ and $Y$ are linearly dependent (as vectors in $k^{n+1}$). The equivalence class of $X$ will be denoted $[x_0:x_1:..:x_n]$.

The topology is given by the closed subsets
\begin{equation}
\V(I) =
\mkset{ [x_0:..x_n] \in \proj^n_k}
      { \forall f\in I \text{ homogeneous}, f(x_0,..x_n) = 0}
\end{equation}
for homogeneous ideals $I$ of the polynomial ring $k[x_0,..x_n]$.
The topology is known as Zariski's topology.
Because $k[x_0,..x_n]$ is a Noetherian ring, every ideal is finitely generated and in particular every homogeneous ideal is finitely generated by homogeneous elements.
So we may say that a closed set consists precisely of the points in projective space where a finite set of homogeneous ideals vanish.

In this framework I want to identify geometric objects with homogeneous ideals of $k[x_0,..x_n]$.
However, most of the time it is easier to speak of the closed subsets of projective space, instead of the ideal $I$, and $I$ itself can be recovered by means of Hilberts's Nullstellensatz in the form of its radical.

A \emph{hypersurface} is given by one equation $f=0$ for $f\in k[x_0,..x_n]$, and its set of points is $\V(f)$.
In case of $f$ being a linear form we call $\V(f)$ a \emph{hyperplane}.

A line is determined by $n-1$ $k$-linearly independent linear forms and a point by $n$ $k$-linearly independent linear forms.



\subsubsection{linear sets}
\begin{todo}
\item define hyperplane, plane, line, point
\item closed subsets yield reduced ideals, but we may attach non-reduced ideals. brought to its full conclusion, we need schemes, but we can do without so far
\end{todo}


\begin{proposition} $n$ hyperplanes in $\proj^n_k$ have at least one common point of intersection.
\end{proposition}
\begin{proof}
To find a point lying on all $n$ hyperplanes amounts to solving a homogeneous system of $n$ linear equations in $n+1$ variables.
Such a system always has at least one non-zero solution.
\end{proof}

Setting $n=2$ an immediate consequence is that two lines in $\proj^2_k$ intersect.
Because lines are intersections of $n-1$ hyperplanes in $\proj^n_k$, we also deduce that lines always intersect with hyperplanes.


So far we have defined a line to be an intersection of hyperplanes, but of course we also want to understand a line as the unique intersection of such hyperplanes containing two distinct points.

Let's fix a projective space $\proj^n_k$ and let $P=[p_0:..p_n]$ and $Q=[q_0:..q_n]$ be distinct points in $\proj^n_k$. Now consider the homomorphism
\begin{equation}
L : \begin{cases}
k[x_0,..x_n] &\to k[\lambda,\mu] \\
f &\mapsto f(\lambda P + \mu Q) := f(\lambda p_0 + \mu q_0, .. \lambda p_n + \mu q_n)
\end{cases}
\end{equation}

Its kernel contains those polynomials, which vanish on $\lambda P + \mu Q$ and in particular on any point $\lambda_0 P + \mu_0 Q$ for $[\lambda_0:\mu_0] \in \proj^1_k$, such as $P$ and $Q$.
These were the points we expected to be contained on the line anyway.
Consider the linear map $\bigoplus_{i=0}^n kx_i \to k\lambda \oplus k\mu$ defined by the $2\times (n+1)$-matrix
\begin{equation}
M=
\begin{pmatrix}
p_0 & \ldots & p_n \\
q_0 & \ldots & q_n
\end{pmatrix}
\end{equation}

Because $P$ and $Q$ are distinct points in projective space, the matrix has full rank 2, and hence the kernel is spanned by $(n-1)$ linear forms $h_0,..h_{n-2}$.
By our choice of the matrix $M$, these linear forms span the kernel of $L$: $h_i(\lambda P + \mu Q) = h_i(P) \lambda + h_i(Q)\mu = 0 \lambda + 0 \mu$.
Dually, $P,Q$ (as vectors) span the solutions of the system of $n-1$ linear equations in $n+1$ variables:
\begin{align}
h_0(x_0,..x_n) =& 0 \\
\vdots& \\
h_{n-2}(x_0,..x_n) =& 0
\end{align}
Hence $\mkset{\lambda P + \mu Q}{\lambda, \mu \in k} = \V(h_0,..h_{n-2})$.

Having seen that two points determine a line we can ask when a line connecting two points of a surface is contained on said surface.
Let me state a very useful fact (given in \cite[section 5.3]{reid1988undergraduate}) before we continue.

\begin{theorem}[Projective Nullstellensatz]
For a set $X$ of points in projective space $\proj^n_k$ define its ideal $\I(X) := \mkset{f \in k[x_0,..x_n]}{f(x_0,..x_n) = 0\,\forall [x_0:..x_n] \in X}$.
Then $\I$ and $\V$ are inclusion reversing and for ideals $J$ with $\V(J) \neq \emptyset$ we have $\I(\V(J)) = \sqrt{J}$.
$\V(J) = \emptyset$ holds iff $J \supset (x_0,..x_n)$.
\end{theorem}

Assume now that $f \in \ker(L)$, then $\V(f) \supset \mkset{\lambda P + \mu Q}{\lambda, \mu \in k} = \V(h_0,..h_{n-2})$ and hence $(f) \subset \sqrt{(f)} \subset \sqrt{(h_0,..h_n)} = (h_0,..h_n)$.
For now take the last equation, which says that an ideal generated by linear forms is radical, as a fact which we will later prove (corollary \ref{corollaryRadical}).
This shows that the homogeneous ideal associated to $\mkset{\lambda P + \mu Q}{\lambda, \mu \in k} \subseteq \proj^n_k$ is indeed generated by the $h_i$
and that the line is contained on $\V(f)$ iff $f(\lambda P + \mu Q) = 0 \in k[\lambda,\mu]$.
