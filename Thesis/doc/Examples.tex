\subsection{Some explicit examples of 27 lines on cubics}


\begin{example}[Clebsch's cubic]
Clebsch's cubic surface $S$ (\cite[§16,p.331 ff.]{clebsch1871ueber}) is given by the intersection of a cubic hypersurface $\V(x^3 + y^3 + z^3 + t^3 + w^3)$ in $\proj^4_k$ with a hyperplane $\V(x+y+z+t+w)$.
To find the lines on this particular cubic is particularly easy due to the symmetry of the defining equations: The group $G := S_5$ operates on $S$ by permuting the homogeneous coordinates, and hence $G$ operates on the set $\mathcal L$ of lines on $S$.
One line is given by $L = \V(x-y,z-t,w)$, and the stabiliser $G_L$ is generated by $(x,y);(z,t);(x,z)(y,t)$.
Applying the orbit-stabiliser theorem we see that the orbit $G(L)$ has cardinality $|G(L)| = \frac{|G|}{|G_L|} = \frac{5!}{2\cdot2\cdot2} = 15$.
Now let $\xi \neq 1$ be a fifth root of unity -- don't forget that our field is algebraically closed!
We now consider a line $L'$ through $P = [\xi^0:\xi^1:\xi^2:\xi^3:\xi^4]$ and its `conjugate' $\overline{P} := [\xi^0:\xi^{-1}:\xi^{-2}:\xi^{-3}:\xi^{-4}] = [\xi^0:\xi^4:\xi^3:\xi^2:\xi^1]$.
The sum of all fifth roots of unity vanishes, as $0 = (\xi^5 - 1) = (\xi - 1)(1 + \xi + \xi^2+\xi^3+\xi^4)$, so the two points lie on $S$.
The sum of the third powers also vanishes: $0 = \sum_{i=0}^4 \xi^{3i \mod 5} \overset{\text{permute the}}{\underset{\text{summands}}=} \sum_{i=0}^4 \xi^i = 0$
Certainly the line connecting the two points lies on the plane $\V(x+y+z+t+w)$, and to show that it also lies on the cubic hypersurface it remains to show for $f = x^3 + y^3 + z^3 + t^3 + w^3$ that $f^{(1)}(P,\overline P) = 0 = f^{(1)}(\overline P,P)$.
\begin{align}
f^{(1)}(P,\overline{P}) =& 3 \sum_{i=0}^4 \xi^{2i}\xi^{-i} = 0 \\
f^{(1)}(\overline{P},P) =& 3 \sum_{i=0}^4 \xi^{-2i}\xi^{i} = 0
\end{align}
Again we can calculate the cardinality of the orbit $G(L')$.
An element $\sigma$ of the stabiliser $G_{L'}$ may only swap $P$ with $\overline{P}$ or multiply the coordinates by some $\xi^i$ (because that doesn't change the point in projective space), hence $|G(L')| = \frac{|G|}{|G_{L'}|} = \frac{5!}{2\cdot 5} = 12$.
Certainly the orbits $G(L')$ and $G(L)$ are disjoint as they have different cardinality and consequently $|G(L') \cup G(L)| = 27$, so we found all of the lines.
\end{example}


\begin{example}[Fermat's cubic]
Fermat's cubic is given by the form $f = x^3 + y^3 + z^3 + t^3$.
A line is given by $\V(x + y, z + t)$, as one can see by elimination of the variables $x$ and $z$.
We can again employ, thanks to the symmetry of $f$, a group operation of $S_4$ on the cubic surface, which permutes the coordinates.
But there is more: Let $\eta$ be the third root of unity, then $\Z3$ operates on a fixed coordinate by multiplying with a power of $\eta$.
For instance $[x:y:z:t]$ is mapped to $[x\eta:y:z:t]$.
We have four such group operations by $\Z3$ (one for each coordinate), so we can say that $G := S_4\times (\Z3)^4$ operates on the lines on the cubic.
One can check that the stabiliser of $L$ is generated by $(x,y);(z,t);(x,z)(y,t) \in S_4$ and $(1,1,0,0),(0,0,1,1) \in (\Z3)^4$, so the orbit of the operation by the product $S_4\times (\Z3)^4$ has cardinality
\begin{equation}
\frac{|G|}{|\langle (x,y);(z,t);(x,z)(y,t) \rangle|\cdot|\langle (1,1,0,0), (0,0,1,1)\rangle|} = \frac{4!\cdot 3^4}{8\cdot 3^2} = 3^3 = 27
\end{equation}
So we see that all of the 27 lines lie in the orbit of $L$.
\end{example}
