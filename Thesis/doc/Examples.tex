\section{Some explicit examples of 27 lines on cubics}

\begin{example}[Fermat's cubic]
Fermat's cubic is given by the polynomial $f = x^3 + y^3 + z^3 + t^3$.
Because our ground field $k$ is assumed to be algebraically closed, let $\eta$ be a third root of unity.
Define for convenience $H(u,v) = \{ u+\eta v, u+\eta^2v, u+v \}$ for polynomials $u,v$.
Then the 27 lines are
$\mkset{ \V(h_1,h_2) }{ (h_1,h_2) \in H(x,y)\times H(z,t) \cup H(x,z)\times H(y,t) \cup H(x,t)\times H(y,z)}$.
\end{example}

\begin{example}[Clebsch's cubic]
Clebsch's cubic surface $S$ (\cite[§16,p.331 ff.]{clebsch1871ueber}) given by the intersection of a cubic hypersurface $\V(x^3 + y^3 + z^3 + t^3 + w^3)$ in $\proj^4_k$ with a hyperplane $\V(x+y+z+t+w)$.
To find the lines on this particular cubic is particularly easy due to the symmetry of the defining equations: The group $G := S_5$ operates on $S$ by permutating the homogeneous coordinates, and hence $G$ operates on the set $\mathcal L$ of lines on $S$.
One line is given by $L = \V(x-y,z-t,w)$, and the stabiliser $G_L$ is generated by $(x,y);(z,t);(x,z)(y,t)$.
Applying the orbit-stabiliser theorem we see that the orbit $G(L)$ has cardinality $|G(L)| = \frac{|G|}{|G_L|} = \frac{5!}{2\cdot2\cdot2} = 15$.
Now let $\xi \neq 1$ be a fifth root of unity -- don't forget that our field is algebraically closed!
We now consider a line $L'$ through $P = [\xi^0:\xi^1:\xi^2:\xi^3:\xi^4]$ and its `conjugate' $\overline{P} := [\xi^0:\xi^{-1}:\xi^{-2}:\xi^{-3}:\xi^{-4}] = [\xi^0:\xi^4:\xi^3:\xi^2:\xi^1]$.
The sum of all fifth roots of unity vanishes, as $0 = (\xi^5 - 1) = (\xi - 1)(1 + \xi + \xi^2+\xi^3+\xi^4)$, so the two points lie on $S$.
The sum of the third powers also vanishes: $0 = \sum_{i=0}^4 \xi^{3i \mod 5} \overset{\text{permute the}}{\underset{\text{summands}}=} \sum_{i=0}^4 \xi^i = 0$
Certainly the line connecting the two points lies on the plane $\V(x+y+z+t+w)$, and to show that it also lies on the cubic hypersurface it remains to show for $f = x^3 + y^3 + z^3 + t^3 + w^3$ that $f^{(1)}(P,\overline P) = 0 = f^{(1)}(\overline P,P)$.
\begin{align}
f^{(1)}(P_1,P_2) =& 3 \sum_{i=0}^4 \xi^{2i}\xi^{-i} = 0 \\
f^{(1)}(P_2,P_1) =& 3 \sum_{i=0}^4 \xi^{-2i}\xi^{i} = 0
\end{align}
Again we can calculate the cardinality of the orbit $G(L')$.
An element $\sigma$ of the stabilizer $G_{L'}$ may only swap $P_1$ with $P_2$ or multiply the coordinates by some $\xi^i$ (because that doesn't change the point in projective space), hence $|G(L')| = \frac{|G|}{|G_{L'}|} = \frac{5!}{2\cdot 5} = 12$.
Certainly the orbits $G(L')$ and $G(L)$ are disjoint as they have different cardinality and consequently $|G(L') \cup G(L)| = 27$, so we found all of the lines.
\end{example}
