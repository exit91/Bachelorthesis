\subsection{Classification of singular cubic curves}

We will study the kinds of singularities which can occur for singular cubic curves.
Let $\V(F)$ be a singular cubic curve.
The first observation is that there exist only one singular point $P = [s_0:s_1:s_2]$ on the curve.
Assume for the contrary that $P \neq P'$ are singular points, then the restriction of $F$ to the line $L = \overline{P,P'}$ has 2 zeroes of multiplicity 2, but $F$ had degree 3 -- a contradiction.
\begin{todo} \item intersection multiplicity not defined yet! \end{todo}
We may assume that this one singular point is $[0:0:1]$ by an appropritate transformation:
By permuting the coordinates we may assume $s_2 \neq 0$ and then the projective transformation $F(x+s_0z,y+s_1z,s_2z)$ has it's singularity at $[0:0:1]$.

One way to examinate the singularity is to consider lines through it. Consider the polynomial $f(ut, vt) \in k[u,v,t]$
where $f := F(x,y,1)$ is the dehomogenisation.
Informally we think of $u,v$ as point $[u:v] \in \proj^1_k$ parametrising all the lines through $[0:0:1]$.
It is apparent that $f$ vanishes on $(0,0)$ and also the partial derivatives vanish as they did for $F$.
We conclude that $f$ has no terms of degree 0 or 1, so $f(ut,vt) = t^2g + t^3h$ where we consider $g,h$ as 2-form and 3-form respectively in $k[u,v]$.
Especially the polynomial $g$ gives us some geometric insight on the `shape' of the singularity.
Before elaborating on that let me state the fundamental theorem of algebra for homogeneous forms:

\begin{lemma}[homogeneous fundamental theorem of algebra] \label{lemmaFundamentalTheorem}
Let $k$ be an algebraically closed field and $g \in k[u,v]$ a $d$-form ($d \in \posnats$).
Then $g$ is a product of $d$ 1-forms.
\end{lemma}
\begin{proof}
Assume that $v$ does not divide $g$ (otherwise we can already factor out $v$).
We can consider $g$ as an element of $k(v)[u]$ and dehomogenize to $g(u/v,1) \in k[u/v]$.
Applying the fundamental theorem of algebra $g(u/v,1) = \alpha\prod_{i=1}^d(u/v - \alpha_i), \quad (\alpha,\alpha_1,..\alpha_d \in k)$ and homogenizing again we get
$g = v^d g(u/v,1) = \alpha\prod_{i=1}^d(u - \alpha_iv)$.
\end{proof}

Let's do a case analysis on $g$.
\paragraph{Case 0: $g=0$.}
In this case we deduce $F \in k[x,y]$ and by lemma \ref{lemmaFundamentalTheorem} the equation $F=0$ defines the union of 3 (not necessarily distinct) lines.
\paragraph{Case 1: $g=h_1h_2$ and $h_1,h_2$ not $k$-linearly dependent.}
We call the singularity a \emph{node}.
\paragraph{Case 2: $g=h^2$}
We call the singularity a \emph{cusp}.

\begin{todo}
\item PIC or didn't happen
\end{todo}


\subsubsection{normal forms of singular cubic curves}

The following discussion on normal forms will follow the argument in \cite[Satz 4.9, p.102]{hulek2000elementare}.
We will obtain in characteristic $\neq 2$ a normal form for cubic curves with nodes and in characteristic $\neq 3$ a normal form for cubic curves with cusps.


\begin{todo}
\item 
\end{todo}
