\subsection{The restriction of equations}

One useful operation one may perform is to take the intersection of a hypersurface $S = \V(f) \subset \proj^n_k$ with a hyperplane $\Pi = \V(h)$.
Now one expects from geometric intuition that the $d$-form $f$ `restricts' to a $d$-form on $\Pi$ where we think of $\Pi$ as $\proj^{n-1}_k$, unless $S$ contains $\Pi$ in which case one would expect the equation to `restrict' to the zero polynomial.
The goal of this section is to make this intuition precise, by defining a homomorphism $\var{res} : k[x_0,..x_n] \to k[x_0,..x_{n-1}]$ and an isomorphism $\proj^{n-1} \overset\sim\to \Pi$.

First we may assume without loss of generality that
$h = \alpha x_n + \widetilde h$ where $0 \neq \alpha \in k$ and $\widetilde h \in k[x_0,..x_{n-1}]$.

Then we define
\begin{equation}
\var{res} : \begin{cases}
x_i \mapsto x_i, & \text{for } i \neq n \\
x_n \mapsto -\frac 1\alpha \widetilde h. &
\end{cases}
\end{equation}

and furthermore we extend it (via the canonical inclusion $k[x_0,..x_{n-1}] \subset k[x_0,..x_n]$) to

\begin{equation}
\widetilde{\var{res}} : k[x_0,.. x_n] \overset{\var{res}}\to k[x_0,.. x_{n-1}] \hookr k[x_0,.. x_n]
\end{equation}

We define the isomorphism $\vartheta : \proj^{n-1}_k \to \Pi$ via
\begin{equation}
\vartheta : [x_0:..x_{n-1}] \mapsto [x_0:..x_{n-1}:-\frac 1\alpha \widetilde h(x_0,..x_{n-1})]
\end{equation}
One can easily see by evaluation that $h$ vanishes on the image of $\vartheta$, so $\vartheta$ maps into $\Pi$. To confirm that we indeed defined an isomorphism, we construct an inverse.
A left-inverse of course has to look like this:
\begin{equation}
\vartheta^{-1} : [x_0:..x_n] \mapsto [x_0:..x_{n-1}]
\end{equation}
To show well-definedness of $\vartheta^{-1}$, one needs to prove that $[x_0:..x_{n-1}] = [0:..0]$ is not in the image. Suppose it were, then $0 = h(x_0,..x_{n-1}) = \alpha x_n + \widetilde h(0)$ and therefore $x_n = -\frac 1\alpha \widetilde h(0) = 0$ as well, so the preimage would have to be $[x_0:..x_n] = [0:..0]$ which cannot happen.
What we've also seen in the calculation so far is that the coordinate $x_n$ depends uniquely on the other ones, hence $\vartheta^{-1}$ as defined above is a right-inverse.

Having everything defined we obtain the following relation of a form $f$ and its restriction to the plane $\var{res}(f)$: Let $P = [p_0:..p_n] \in \Pi$ be a point on the plane, so $p_n = -\frac 1\alpha \widetilde h(p_0,..p_{n-1}) = -\frac 1\alpha \widetilde h(\vartheta^{-1}(P))$.

\begin{align}
f(P) = 0
&\text{ iff } f(p_0,..p_{n-1},-\frac 1\alpha \widetilde h(\vartheta^{-1}(P))) = 0
\\
&\text { iff } \var{res}(f)(\vartheta^{-1}(P)) = 0
\\
(&\text{ iff } \widetilde{\var{res}}(f)(P) = 0)
\end{align}

This allows us to understand the projective variety $\Pi \cap \V(f)$ in terms of the equation $\var{res}(f) = 0$ by $\Pi \cap \V(f) \simeq \V(\var{res}(f)) \subset \proj^{n-1}_k$.
Iterating this process we may consider further restrictions to $\proj^{n-2}_k$ etc. (we will make this statement precise in corollary \ref{corollaryRadical}).
Thus it makes sense to talk about restricting a form to a linear subspace.

Another thing I want to point out is that the endomorphism $\widetilde{\var{res}}$ is idempotent (i.e. $\widetilde{\var{res}}.\widetilde{\var{res}} = \widetilde{\var{res}}$) and the kernel is the ideal generated by $h$: On one hand $\var{res}(h) = 0$, on the other hand if $\var{res}$ maps a form $f$ to $0$, then $f$ vanishes on $\Pi$, so $h$ divides $f$ (say, by Hilbert's Nullstellensatz).

In particular we obtain
\begin{proposition} \label{propositionRestriction}
Let $k$ be algebraically closed.
For any $d$-form $f \in k[x_0,..x_n]$ and $\widetilde{\var{res}}$ as defined before there exists a ($d-1$)-form $r$ such that
\begin{equation}
f = \widetilde{\var{res}}(f) +  hr
\end{equation}
In particular, if $h$ does not divide $f$, then $\widetilde{\var{res}}(f)$ is non-zero and has degree $\deg(f)$.
\end{proposition}

\begin{remark}
Viewing a variety as a system of polynomial equations, what we have done by restricting to a hyperplane is to `eliminate' the variable $x_n$ from the equations.
Henceforth we will talk of eliminating a variable to specify which variable serves the role, which $x_n$ took in this section.
\end{remark}

\begin{definition}
We call $\widetilde{\var{res}(f)}$ the `restriction of the equation of the hypersurface $\V(f)$' (even though, technically, $f$ is not an equation, but $f=0$ is)
 or the equation $f$ with the variable $x_n$ eliminated.
\end{definition}

%%We can formulate the result in a slightly higher level of generality. So first we replace algebraic subsets of $\proj^n_k$ by ideals $J \subset k[x_0,..x_n]$.
%%Furthermore, algebraic subsets of $J = \V(h)$ correspond to ideals $I \subset k[x_0,..x_n]/J$.

\begin{corollary} \label{corollaryRadical}
Let $h_1,.. h_m \in k[x_0,..x_n]$ be $m$ $k$-linearly independent linear forms.
Then there exists an isomorphism $k[x_0,..x_n]/(h_1,..h_m) \simeq k[x_0,..x_{n-m}]$.
Furthermore the ideal $(h_1,..h_m)\subset k[x_0,..x_n]$ is prime.
\end{corollary}
\begin{proof}
We prove the first assertion.
The case $m=1$ has been covered already.
The inductive step goes as follows:
Let $I = (h_1,..h_{m-1}), I+(h_m) = (h_1,..h_m)$ be ideals of $k[x_0,..x_n]$ and assume without loss of generality that the $x_n$ coefficient of $h_m$ is non-zero.
The restriction homomorphism for the hyperplane $\V(h_m)$ gives us a short exact sequence of $k[x_0,..x_n]$-modules
\begin{equation}
0 \to (h_m) \overset{\ker(\var{res})}\to k[x_0,..x_n] \overset{\var{res}}\to k[x_0,..x_{n-1}] \to 0
\end{equation}
Because $\var{res}$ is an epimorphism, we get isomorphisms
\begin{equation}
\frac{k[x_0,..x_{n-1}]}{\var{res}((h_m))}
\simeq \frac{\var{res}^{-1}(k[x_0,..x_{n-1}])}{\var{res}^{-1}(\var{res}((h_m))}
= \frac{k[x_0,..x_n]}{I+(h_m)}
= \frac{k[x_0,..x_n]}{(h_1,..h_m)}.
\end{equation}

Because the linear forms are $k$-linear independent, none of the forms $h_0,..h_{m-1}$ lie in the kernel $I$ of the restriction homomorphism.
Obviously the restriction homomorphism maps the linear forms $h_i$ to linear forms $\var{res}(h_i) \neq 0$ which are linearly independent.
Too see this, choose $\beta_i \in k$ such that $h_i = \var{res}(h_i) + \beta_i h_m$.
Any non-trivial linear combination $0 = \sum_{i=0}^{m-1} \alpha_i \var{res}(h_i)$ of the $\var{res}(h_i)$, induces a non-trivial linear combination $0 = \sum_{i=0}^{m-1} \alpha_i(h_i - \beta_i h_m) = \sum_{i=0}^{m-1}\alpha_i h_i - (\sum_{i=0}^{m-1} \alpha_i\beta_i)h_m$ of the $h_i$.
Hence we can apply the induction hypothesis on $\var{res}(I) = (\var{res}(h_0),..\var{res}(h_{m-1})) \subset k[x_0,..x_{n-1}]$, which concludes our proof.


The second claim is an easy corollary, as $k[x_0,..x_n]/(h_0,..h_m) \simeq k[x_0,..x_{n-m}]$ is an integral domain which is equivalent to the ideal $(h_0,..h_m)$ being prime.

\end{proof}

I want to finish this section with a more geometrical observation.
\begin{proposition} \label{propositionDegreeOfSurface}
Let $\V(f) \subset \proj^n_k$ be a hypersurface with $f$ having degree $d> 0$.
Either $\V(f)$ intersects a line $L$ in at most $d$ points or $\V(f)$ contains that line.
Also $L$ intersects $\V(f)$ at least once!
\end{proposition}

Before giving the proof, let me state the fundamental theorem of algebra for homogeneous forms:

\begin{lemma}[homogeneous fundamental theorem of algebra] \label{lemmaFundamentalTheorem}
Let $k$ be an algebraically closed field and $g \in k[u,v]$ a $d$-form ($d \in \posnats$).
Then $g$ factors into a product of $d$ linear forms.
\end{lemma}
\begin{proof}
Assume that $v$ does not divide $g$ (otherwise we can already factor out $v$).
We can consider $g$ as an element of $k(v)[u]$ and dehomogenise to $g(u/v,1) \in k[u/v]$.
Applying the fundamental theorem of algebra $g(u/v,1) = \alpha\prod_{i=1}^d(u/v - \alpha_i), \quad (\alpha,\alpha_1,..\alpha_d \in k)$ and homogenising again we get
$g = v^d g(u/v,1) = \alpha\prod_{i=1}^d(u - \alpha_iv)$.
\end{proof}

\begin{proof}[Proof of Proposition \ref{propositionDegreeOfSurface}]
Restricting $f$ to $L$ we obtain a $d$-form or 0.
In the former case, lemma \ref{lemmaFundamentalTheorem} tells us that the restriction has at most $d$ zeroes and at least one, meaning that the line has at most $d$ and at least one point of intersection.
The latter case implies that $f$ vanishes on $L$, so $L \subset \V(f)$.
\end{proof}
