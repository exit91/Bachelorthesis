\subsection{the group action of $PGL_n$ on $\proj^n_k$}

The projective space $\proj^n_k$ is naturally endowed with a group action, the group being the group of automorphisms on $\proj^n_k$.
We can give this group explicitly as the automorphisms $f$ given by linear forms $f_i \in k[x_0,..x_n]$.
\begin{equation}
f(x_0,..x_n) = (f_0(x_0,..x_n),..f_n(x_0,..x_n))
\end{equation}
(A proof of this fact is given in \cite[example 7.1.1]{hartshorne1977algebraic} and involves the functoriality of the Picard group, which we will not cover in this thesis.)
Such automorphisms correspond precisely to invertible $(n+1)\times(n+1)$-matrices modulo non-zero scalar multiples, consisting of the coefficients of the $f_i$.
We call an element of the automorphism group a \emph{projective transformation}.

We will study geometric objects, i.e. projective subsets of some $\proj^n_k$, up to projective automorphism.
That means that we have to establish some properties which are invariant under projective transformation, but also we need to have `enough' projective transformations to work with in a meaningful way.
The following theorem makes this precise.

\begin{theorem} \label{theoremGroupAction}
$PGL_n$ operates transitively on the set $C \subset (\proj^n_k)^{n+2}$ of $(n+2)$-tuples of points, for which no $n+1$ lie on a hyperplane.
\end{theorem}
\begin{proof}
Let's fix some notation first. Let $\{ e_i \}_0^n$ be standard basis of $k^{n+1}$, $e_0 = (1,0,..)^T, e_1 = (0,1,0,..)^T$ etc.
Additionally define $e_{n+1} := \sum_{i=0}^n e_i$.
We can consider such vectors as points of $\proj^n_k$ and write them in brackets, e.g. $[e_0] = [1:0:..0]$.
It suffices to show the following claim: The $(n+2)$-tuple $([e_0],..[e_n],[e_{n+1}])$ lies in every orbit.

Let $([P_0],..[P_{n+1}]) \in C$ be $(n+2)$ points, represented by the vectors $P_i \in k^{n+1}$.
Because there is no canonical choice for the $P_i$, we can also choose $\lambda_i P_i$ as representative of $[P_i]$ for some $\lambda_i \in k^\times$ to be determined later.
We write $\underline \lambda  := (\lambda_0,..\lambda_n)$ for any choice of the $\lambda_i,\,(i\leq n)$ and $\underline 1 := (1,1,..1)$.
The condition that no $(n+1)$ of the points lie on a plane implies that the matrix
\begin{equation}
N_{\underline\lambda}
=
\begin{pmatrix}
\textemdash & \lambda_0 P_0^T & \textemdash \\
& \vdots &  \\
\textemdash & \lambda_n P_n^T & \textemdash \\
\end{pmatrix}
\end{equation}
has full rank and is hence invertible with some inverse $M_{\underline\lambda}$
Note that for any choice of $\underline\lambda$ the matrix $M_{\underline\lambda}^T$ maps $\lambda_iP_i$ to $e_i$ for $i \leq  n$.
We wish to find a choice of $\underline\lambda$, such that $P_{n+1}$ is mapped to $e_{n+1}$:
\begin{align}
& M_{\underline\lambda}^TP_{n+1} = e_{n+1} \\
\Leftrightarrow & P_{n+1} = N_{\underline\lambda}^T e_{n+1} \\
\Leftrightarrow & P_{n+1} = N_{\underline 1}^T \diag(\lambda_0,..\lambda_n) e_{n+1}
                          = N_{\underline 1}^T (\lambda_0,..\lambda_n)^T \\
\Leftrightarrow & M_{\underline 1}^T P_{n+1} = (\lambda_0,..\lambda_n)^T
\end{align}
Clearly, for this choice of $\underline\lambda$, $M_{\underline \lambda}^T$ defines a projective transformation mapping each $[P_i]$ to $[e_i]$, which shows our claim.
\end{proof}

\begin{lemma} \label{lemmaTransformOfLinearSpaces}
A projective transformation maps linear spaces to linear spaces.
\end{lemma}
\begin{proof}
It suffices to show that hyperplanes are mapped to hyperplanes.
Let $h \in k[x_0,..x_n]$ be a linear form and $\phi$ a projective transformation on $\proj^n_k$.
\begin{equation}
P \in \phi(\V(h)) \Leftrightarrow \phi^{-1}(P) \in \V(h) \Leftrightarrow (h.\phi^{-1})(P) = 0 \Leftrightarrow P \in \V(h.\phi^{-1})
\end{equation}
Clearly $h.\phi^{-1}$ is a linear form and non-zero, as $h.\phi^{-1}.\phi = h \neq 0$.
\end{proof}


\begin{corollary}
In the projective plane $\proj^2_k$, consider 3 pairwise distinct points $L_0,L_1,L_2$.
The line $L_2$ intersects the union $L_0 \cup L_1$ in either 1 or 2 points.
Depending on this distinction, the lines can be projectively transformed to $\V(x_0),\V(x_1),\V(x_0-x_1)$ or $\V(x_0),\V(x_1),\V(x_2)$.
\end{corollary}
\begin{proof}
By the previous lemma, a lines are being mapped to lines (one can see this either by considering how the hyperplanes which intersect in the line transform or just by a dimension argument).
Consider now the first case, $L_0,L_1,L_2$ intersecting all in one point $O$.
Let $P_i$ each be a point on $L_i$ different from $O$.
Using theorem \ref{theoremGroupAction} there is a projective transformation mapping $O, P_0,P_1,P_2$ to $[0:0:1],[1:0:1],[0:1:1],[1:1:1]$.
This maps $L_0 = \overline{O,P_0}, L_1 = \overline{O,P_1}, L_2 = \overline{O,P_2}$ to $\V(x_1),\V(x_0),\V(x_0 - x_1)$.

In the second case $L_0,L_1,L_2$ have no common intersection, and $L_0 \cap L_1 = A, L_0 \cap L_2 = B, L_1 \cap L_2 = C$ are distinct.
Now map $A,B,C$ to $[0:0:1],[0:1:0],[1:0:0]$, which maps $L_0 = \overline{A,B}, L_1=\overline{A,C}, L_2=\overline{B,C}$ to $\V(x_0),\V(x_1),\V(x_2)$.
\end{proof}

\begin{corollary} \label{corollaryTransformPlaneWithPointOnIt}
Let $H$ be a plane in $\proj^3_k$ and $P$ be a point on it.
Then there exists a projective transformation sending the plane to $V(x_3)$ and the point to $[0:0:1:0]$.
\end{corollary}
\begin{proof}
The plane is determined by three points, $P, P_1,P_2$.
A projective transformation sends them to $[0:0:1:0], [0:1:0:0],[1:0:0:0]$ and along with it the plane $H$ to $\V(x_3)$.
\end{proof}

\begin{proposition} \label{propositionLiftingAutomorphisms}
Suppose $\iota : \proj^n_k \hookr \proj^N_k$ ($n \leq N$) is an inclusion of projective varieties defined in terms of linear forms and suppose $\phi$ is a projective transformation on $\iota$.
Then there exists a lift to a projective transformation $\overline\phi$ on $\proj^N_k$, such that the diagram
\begin{equation}
\begin{xy}
(0,20)*+{\proj^n_k}="n";
(30,20)*+{\proj^N_k}="N";
(0,0)*+{\proj^n_k}="n1";
(30,0)*+{\proj^N_k}="N1";
{\ar@{^{(}->} "n";"N"}?*!/_2mm/{\iota};
{\ar@{^{(}->} "n1";"N1"}?*!/_2mm/{\iota};
{\ar@{->} "n";"n1"}?*!/_2mm/{\phi};
{\ar@{->} "N";"N1"}?*!/_2mm/{\overline{\phi}};
\end{xy}
\end{equation}
commutes.
\end{proposition}
\begin{proof}
For $n=N$ define $\overline\phi = \iota.\phi.\iota^{-1}$.
If $n < N$ it suffices to prove this result for $N = n+1$.
Let's say $\iota$ is given by $\iota = (h_0,..h_n,0)$, the $h_i$ being linear forms, and because $\iota$ is injective, the $h_i$ span the vector space of all linear forms in $k[x_0,..x_n]$.
It follows, that there exists $g_0,..g_n$ linear forms, defining a projective transformation $\iota'$ such that $\iota' = (g_0,..g_n,x_{n+1})$ (hence $(\iota')^{-1} = (h_0,..h_n,x_{n+1})$) and $\iota'.\iota = (x_0,..x_n,0)$.
We now can define $\overline{\phi'} = (\phi,x_{n+1})$ and finally $\overline\phi = (\iota')^{-1}.\overline{\phi'}.\iota'$.
By construction $\iota.\phi = \overline\phi . \iota$ holds.
\end{proof}


Finishing this section I want to list some properties which are stable under projective transformation.
Some of the assertions have been shown, some notions would need some explanation to be found in any down-to-earth textbook on algebraic geometry (\cite{shafarevich1994basic, harris1992algebraic, brieskorn2012plane}, etc.)
\begin{itemize}
\item degree of a hypersurface
\item linear subspaces, see lemma \ref{lemmaTransformOfLinearSpaces}
\item irreducibility
\item Krull dimension
\item incidence
\item tangent planes and singularities are well-behaved under projective transformation, see proposition \ref{propositionTangentTransform}
\item the hessian hypersurface is well-behaved, i.e. for $S$ a hypersurface and $H$ its hessian hypersurface one can show $\phi(S \cap H) = \phi(S) \cap \phi(H)$.
\item any morphism projective varieties maps closed sets to closed sets
\end{itemize}

