\subsection{the group action of $PGL_n$ on $\proj^n_k$}

The projective space $\proj^n_k$ is naturally endowed with a group action, the group being the group of automorphisms on $\proj^n_k$.
We can give this group explicitly as the automorphisms $f$ given by linear forms $f_i \in k[x_0,..x_n]$.
\begin{equation}
f(x_0,..x_n) = (f_0(x_0,..x_n),..f_n(x_0,..x_n))
\end{equation}
(A proof of this fact is given in \cite[example 7.1.1]{hartshorne1977algebraic} and involves the functoriality of the Picard group, which we will not cover in this thesis.)
Such automorphisms correspond precisely to invertible $(n+1)\times(n+1)$-matrices modulo non-zero scalar multiples, consisting of the coefficients of the $f_i$.
We call an element of the automorphism group a \emph{projective transformation}.

We will study geometric objects, i.e. projective subsets of some $\proj^n_k$, up to projective automorphism.
That means that we have to establish some properties which are invariant under projective transformation, but also we need to have `enough' projective transformations to work with in a meaningful way.
The following theorem makes this precise.

\begin{theorem}
$PGL_n$ operates transitively on the set $C \subset (\proj^n_k)^{n+2}$ of $(n+2)$-tuples of points, for which no $n+1$ lie on a hyperplane.
\end{theorem}
\begin{proof}
Let's fix some notation first. Let $\{ e_i \}_0^n$ be standard basis of $k^{n+1}$, $e_0 = (1,0,..)^T, e_1 = (0,1,0,..)^T$ etc.
Additionally define $e_{n+1} := \sum_{i=0}^n e_i$.
We can consider such vectors as points of $\proj^n_k$ and write them in brackets, e.g. $[e_0] = [1:0:..0]$.
It suffices to show the following claim: The $(n+2)$-tuple $([e_0],..[e_n],[e_{n+1}])$ lies in every orbit.

Let $([P_0],..[P_{n+1}]) \in C$ be $(n+2)$ points, represented by the vectors $P_i \in k^{n+1}$.
Because there is no canonical choice for the $P_i$, we can also choose $\lambda_i P_i$ as representative of $[P_i]$ for some $\lambda_i \in k^\times$ to be determined later.
We write $\underline \lambda  := (\lambda_0,..\lambda_n)$ for any choice of the $\lambda_i,\,(i\leq n)$ and $\underline 1 := (1,1,..1)$.
The condition that no $(n+1)$ of the points lie on a plane implies that the matrix
\begin{equation}
N_{\underline\lambda}
=
\begin{pmatrix}
\textemdash & \lambda_0 P_0^T & \textemdash \\
& \vdots &  \\
\textemdash & \lambda_n P_n^T & \textemdash \\
\end{pmatrix}
\end{equation}
has full rank and is hence invertible with some inverse $M_{\underline\lambda}$
Note that for any choice of $\underline\lambda$ the matrix $M_{\underline\lambda}^T$ maps $\lambda_iP_i$ to $e_i$ for $i \leq  n$.
We wish to find a choice of $\underline\lambda$, such that $P_{n+1}$ is mapped to $e_{n+1}$:
\begin{align}
& M_{\underline\lambda}^TP_{n+1} = e_{n+1} \\
\Leftrightarrow & P_{n+1} = N_{\underline\lambda}^T e_{n+1} \\
\Leftrightarrow & P_{n+1} = N_{\underline 1}^T \diag(\lambda_0,..\lambda_n) e_{n+1}
                          = N_{\underline 1}^T (\lambda_0,..\lambda_n)^T \\
\Leftrightarrow & M_{\underline 1}^T P_{n+1} = (\lambda_0,..\lambda_n)^T
\end{align}
Clearly, for this choice of $\underline\lambda$, $M_{\underline \lambda}^T$ defines a projective transformation mapping each $[P_i]$ to $[e_i]$, which shows our claim.

\end{proof}


\begin{todo}
\item define projective equivalence
\item theory of lines need to be discussed before this
\end{todo}
\begin{corollary}
In the projective plane $\proj^2_k$, consider 3 pairwise distinct points $L_0,L_1,L_2$.
The line $L_2$ intersects the union $L_0 \cup L_1$ in either 1 or 2 points.
Depending on this distinction, the lines can be projectively transformed to $V(x_0),V(x_1),V(x_0-x_1)$ or $V(x_0),V(x_1),V(x_2)$.
\end{corollary}
\begin{proof}
% TODO
\end{proof}

\begin{corollary} \label{corollaryTransformPlaneWithPointOnIt}
Let $H$ be a plane in $\proj^3_k$ and $P$ be a point on it.
Then there exists a projective transformation sending the plane to $V(x_3)$ and the point to $[0:0:0:1]$.
\end{corollary}
\begin{proof}
% TODO, brauche ich das?
\end{proof}

\subsubsection{lifting projective transformations}

\begin{todo}
\item can we lift projective transformations on the plane to the space, say, given an inclusion of the plane to the space etc..
\end{todo}

\begin{proposition} \label{propositionLiftingAutomorphisms}
% TODO
\end{proposition}

\subsubsection{properties stable under projective transformation}
\begin{todo}
\item degree
\item irreducibility
\item Krull dimension
\item X contained in Y relation
\item X intersects Y relation
\item A is union of B and C relation
\item topology
\item tangent planes and singularities are well-behaved under projective transformation \ref{propositionTangentTransform}
\item inflexion points
\item generators (x,y,z) -> (fx,fy,fz)
\end{todo}

