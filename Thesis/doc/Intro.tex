\section{Introduction}

In 1849 Arthur Cayley made the remarkable discovery that there are finitely many lines on a `general' cubic surface -- these are surfaces defined by zeroes of a homogeneous polynomial of degree three in four variables.
George Salmon then set out to count these lines and even more surprisingly it turned out that a cubic surface contains 27 lines in general \cite[p. 496]{salmon1882treatise}.
In fact, if we go a degree down, every quadric surface contains infinitely many lines and going degrees up, almost all (in a sort of measure theoretic sense) surfaces of degree higher than three contain no lines\footnote{a proof can be found in \cite[theorem 1.27]{shafarevich1994basic}}.

This thesis will elaborate on the details of the proof given in \cite[§7]{reid1988undergraduate} and introduce the geometric facts in a rather self-contained fashion.
