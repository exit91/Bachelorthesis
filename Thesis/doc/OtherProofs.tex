\section{Other proofs}

In his textbook `Basic algebraic geometry' Shafarevich\footnote{Shafarevich's student Dolgachev gives the same proof in less detail in \cite[theorem 9.1.13]{dolgachev2012classical} as well} proves the existence of a line on a general cubic surface by relating both kinds of geometric objects in the following way:
We can consider lines as points in some projective space $\mathcal L$ (a so called Grassmannian) and likewise we can consider cubic surfaces as points in a projective space $\mathcal S$: Just take the coefficients of the defining polynomial as homogeneous coordinates.
One can then show that there exists a polynomial relation between coefficients of a line and a surface, determining whether the line lies on the surface.
This means that one can define define a projective variety $\Gamma \subset \mathcal L \times \mathcal S$, such that $(L,S) \in \Gamma$ iff the line corresponding to $L$ lies in the surface corresponding to $S$.
Now, a surface corresponding to $S$ contains a line iff the fibre $\psi^{-1}(S)$ of the projection map $\psi : \mathcal L \times \mathcal S \to \mathcal S$ is non-empty (or equivalently iff $\psi(\Gamma) = \mathcal S$).
The non-emptyness of the fibre then is shown by the `theorem on the dimension of fibres', which is some sort of analogue to the dimension theorem of linear algebra.

Dolgachev gives a symmetric proof for counting the 27 lines, based on \ref{corollaryFivePlanes} in \cite[theorem 9.1.13]{dolgachev2012classical}.
We can sketch his argument like this: having found three lines $L_0,L_1,L_2 = L,\Lambda^1(L),\Lambda'^1(L)$, he considers the planes $\Pi^j(L_i)$ (further remark: these planes are called tritangent planes in the literature) which give for each $L_i$ four additional tritangent planes, each containing two new lines.
Dolgachev then proceeds to show that the $3+8+8+8 = 27$ are in fact all the lines.

Finally there is a proof given in Hartshorne's standard reference `Algebraic Geometry'.
The proof \cite[theorem 4.9]{hartshorne1977algebraic} goes down a whole different path.
He views a cubic surface as a `blowup' of six points in a plane.
At this point we won't elaborate on the details, but we will point out, that this point of view has been used for some profit to generate beautiful cubic surfaces and lines on them on a computer (\cite{van2003visual}).
