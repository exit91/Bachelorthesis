\subsection{plane conics}

\subsubsection{the question of degeneracy}

A plane conic is a algebraic variety $\V(g)$ given by a quadratic form $g \in k[x_0,x_1,x_2]$. One might ask the question whether the conic is a union of two lines (in which case the conic is called \emph{degenerate}), or in algebraic terms, whether $g$ factors into two linear forms or whether it is irreducible.
Let's turn our attention the an easier question: When is a conic singular?

Assume that the characteristic of our base field $k$ is not 2, then the conic can be written, for appropriate coefficients $a,b,c,d,e,f \in k$ as:
\begin{equation}
g = ax_0^2 + bx_0x_1 + cx_1^2 + dx_0x_2 + ex_1x_2 + fx_2^2
\end{equation}
The singular points are given by the system of equations
\begin{equation}
\del_{x_0} g = \del_{x_1} g = \del_{x_2} g = 0
\end{equation}
which written out in matrix notation, amounts to
\begin{align}
\underset{=:M}{\underbrace{
\begin{pmatrix}
2a & b & e \\
b & 2c & d \\
e & d & 2f
\end{pmatrix}
}}
\begin{pmatrix}
x_0 \\ x_1 \\ x_2
\end{pmatrix}
=
0
\end{align}
We call the matrix $M$.
A singular point $[s_0:s_1:s_2] \in \proj^2_k$ would of course be a non-zero solution of above equation and as such can only exist precisely if the determinant of $M$ vanishes.
So far we have obtained

\begin{corollary}
Let $k$ be a field of characteristic not 2 and $g =  ax_0^2 + bx_0x_1 + cx_1^2 + dx_0x_2 + ex_1x_2 + fx_2^2
\in k[x_0,x_1,x_2]$ be a quadratic form. The conic $\V(g) \subset \proj^2_k$ is singular if and only if
\begin{equation}
\frac 12
\det
\begin{pmatrix}
2a & b & e \\
b & 2c & d \\
e & d & 2f
\end{pmatrix}
= 4acf + bde - ce^2 - ad^2 - b^2f = 0
\end{equation}
\end{corollary}

Incidentally this statement holds true for characteristic 2 as well, even though we need to approach the proof a little differently.
At this point I remind the reader that addition and subtraction are the same in characteristic 2, in the sense that negation operation is just the identity, which will simplify the calculations a little.
The corollary then translates to $g$ having a singular point if and only if $bde + ce^2 +ad^2 + b^2f = 0$.
The set of singular points is by definition the intersection of $V = \V(\del_{x_0}g,\del_{x_1}g,\del_{x_2}g)$ and $\V(g)$, so let's calculate the points in the first variety: Assuming that not all coefficients $b,f,e$ vanish, the only point on $V$ is $[e:d:b]$, as can be seen by Gaussian elimination where we distinguish the cases of $0,1$ or $2$ of the coefficients $b,d,e$ vanishing.
This point $[e:d:b]$ lies on $\V(g)$ iff $0 = g(e,d,b) = ae^2 + bde + cd^2 + dbe + ebd + fb^2 = bde + ae^2 + cd^2 + fb^2$ as desired.
If all of $b,d,e$ vanish, then $V$ is the whole space and every point of $\V(g)$ is singular ($\V(g) = \V((\sqrt{a}x_0 + \sqrt{c}x_1 + \sqrt{f}x_2)^2)$ being a doubled line) and also $bde + ce^2 + ad^2 + b^2f = 0$. We have proven:

\begin{corollary}
Let $k$ be a field of any characteristic and $g = ax_0^2 + bx_0x_1 + cx_1^2 + dx_0x_2 + ex_1x_2 + fx_2^2
 \in k[x_0,x_1,x_2]$ be a quadratic form. The conic $\V(g) \subset \proj^2_k$ is singular if and only if $4acf + bde - ce^2 - ad^2 - b^2f = 0$.
\end{corollary}


Returning to our initial question we want to establish the fact that the conic given by the quadratic form $g$ is irreducible if and only if it is non-singular.
For that assume reducibility, that is $g = h_1h_2$ for 1-forms $h_1$ and $h_2$.
Then $\del_{x_i}g = h_1\alpha + \beta h_2$ for $\alpha := \del_{x_i}h_2 \in k$ and $\beta := \del_{x_i}h_2 \in k$.
Assume now that a point $P$ lies in the intersection of $\V(h_1)$ and $V(h_2)$, then $\del_{x_i}g(P) = h_1(P)\alpha + \beta h_2(P) = 0$, so the intersection is a singularity.
Of course, in the projective plane there always exists an intersection for any two lines.
(This is just a consequence of linear algebra: Say $h_1 = a_0x_0 + a_1x_1 + a_2x_2, h_2 = b_0x_0 + b_1x_1 + b_2x_2$, then the kernel of the matrix $\begin{pmatrix} a_0 & a_1 & a_2 \\ b_0 & b_1 & b_2 \end{pmatrix}$ is nontrivial and if $(s_0,s_1,s_2) \neq 0 \in k^3$ lies in the kernel, then $P := [s_0:s_1:s_2]$ lies in the intersection.)

The converse can be seen as follows. Let $P=[p_0:p_1:p_2]$ be a singularity and $P'=[p'_0:p'_1:p'_2]$ is any other point on the conic (for instance any intersection point of $\V(g) \cap \V(x_i)$).
For $g$ to contain the line through $P$ and $P'$ means that $g(\lambda P + \mu P') = 0 \in k[\lambda,\mu]$.

Again, Euler's equality shows itself to be quite useful in the calculation
\begin{align}
0
= 2f(\lambda P')
=& \sum_{i=0}^2 \lambda p'_i \del_{x_i}g(\lambda P') 
+ \sum_{i=0}^2 \lambda p'_i \underset{=0}{\underbrace{\del_{x_i}g(\mu P)}}
\\
\overset{\del_{x_i}g\text{ is linear }}=& \sum_{i=0}^2 \lambda p'_i \del_{x_i} g(\lambda P'+\mu P) 
\\
=& 2g(\lambda P' + \mu P) - \sum_{i=0}^2 \mu p_i \del_{x_i}g(\lambda P' + \mu P)
\end{align}

Finally I claim that the last sum disappears due to the equality
\begin{equation}
\sum_{i=0}^2 p_i \del_{x_i}g = 0 \in k[x_0,x_1,x_2]
\end{equation}

The calculation is straight-forward:
\begin{align}
\sum_{i=0}^2 p_i \del_{x_i} g
=& \sum_{i=0}^2 p_i \sum_{j=0}^2 (\del_{x_j}\del_{x_i}g)x_j
\\
\overset{\text{rearrange}}=& \sum_{j=0}^2 x_j \sum_{i=0}^2 (\del_{x_i}\del_{x_j}g)p_i = \sum_{j=0}^2 x_j \del_{x_j}g(P) = 0
\end{align}

Now that we have shown the sum to disappear, we obtain $0 = 2g(\lambda P' + \mu P)$ (in characteristic not 2), hence the conic contains a line.

For characteristic 2 I give a separate proof, as Euler's formula does not lead anywhere:
For $g$ to be singular at $P$ means that the following relations hold:
\begin{align}
bp_1 + dp_2 =& 0\\
bp_0 + ep_2 =& 0\\
dp_0 + ep_1 =& 0
\end{align}
With this we may write out $g(\lambda P + \mu P')$:
\begin{align}
a(\lambda p_0 + \mu p'_0)^2
+c(\lambda p_1 + \mu p'_1)^2
+f(\lambda p_2 + \mu p'_2)^2 \\
+b (\lambda p_0 + \mu p'_0)(\lambda p_1 + \mu p'_1)
+d (\lambda p_0 + \mu p'_0)(\lambda p_2 + \mu p'_2)
+e (\lambda p_1 + \mu p'_1)(\lambda p_2 + \mu p'_2)
\end{align}

Upon expanding we may collect the $\mu\lambda$ terms and see that they vanish after applying the previous three relations.
Furthermore the quadratic terms we have the equality (by means of the Frobenius homomorphism)
$a(\lambda p_0 + \mu p'_0)^2 = a\lambda^2p_0^2 + a \mu^2 {p'}_0^2$ etc.
It turns out, that after having done these transformations we get the equality $P(\lambda P + \mu P') = P(\lambda P) + P(\mu P') = 0$ as desired.
All in all we have shown:

\begin{theorem}
Let $k$ be a field and $\V(g) \subset \proj^2_k$ a conic given by a quadratic form $g = ax_0^2 + bx_0x_1 + cx_1^2 + dx_0x_2 + ex_1x_2 + fx_2^2$.
Then the following are equivalent:
\begin{enumerate}
\item The conic is degenerate.
\item The quadratic form $g$ factors into two linear forms, $g=h_1h_2$.
\item $\V(g)$ is a union of two lines.
\item $\V(g)$ is singular.
\item $4acf + bde - ce^2 - ad^2 - b^2f = 0$
\end{enumerate}
\end{theorem}


\begin{remark}
Otto Hesse has shown that over $k=\complex$ a curve $\V(g)$ decomposes into lines iff the Hessian curve $\V(\det(\del_{x_i}\del_{x_j}h))$ lies in $\V(g)$ (\cite[p.289]{brieskorn2012plane}), however we've seen that this result does not hold over arbitrary fields.
The equation of the Hessian curve in case of the conic considered in this section is precisely $2(4acf + bde - ce^2 - ad^2 - b^2f) = 0$.
\end{remark}


\subsubsection{the two rulings of a nonsingular plane conic}

% TODO


